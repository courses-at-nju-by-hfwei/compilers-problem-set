% hw2.tex

% !TEX program = xelatex
%%%%%%%%%%%%%%%%%%%%
% see http://mirrors.concertpass.com/tex-archive/macros/latex/contrib/tufte-latex/sample-handout.pdf
% for how to use tufte-handout
\documentclass[a4paper, justified]{tufte-handout}

% hw-preamble.tex

% geometry for A4 paper
% See https://tex.stackexchange.com/a/119912/23098
\geometry{
  left=20.0mm,
  top=20.0mm,
  bottom=20.0mm,
  textwidth=130mm, % main text block
  marginparsep=5.0mm, % gutter between main text block and margin notes
  marginparwidth=50.0mm % width of margin notes
}

% for colors
\usepackage{xcolor} % usage: \color{red}{text}
% predefined colors
\newcommand{\red}[1]{\textcolor{red}{#1}} % usage: \red{text}
\newcommand{\blue}[1]{\textcolor{blue}{#1}}
\newcommand{\teal}[1]{\textcolor{teal}{#1}}

\usepackage{todonotes}

% heading
\usepackage{sectsty}
\setcounter{secnumdepth}{2}
\allsectionsfont{\centering\huge\rmfamily}

% for Chinese
\usepackage{xeCJK}
\usepackage{zhnumber}
\setCJKmainfont[BoldFont=FandolSong-Bold.otf]{FandolSong-Regular.otf}

% for fonts
\usepackage{fontspec}
\newcommand{\song}{\CJKfamily{song}}
\newcommand{\kai}{\CJKfamily{kai}}

% To fix the ``MakeTextLowerCase'' bug:
% See https://github.com/Tufte-LaTeX/tufte-latex/issues/64#issuecomment-78572017
% Set up the spacing using fontspec features
\renewcommand\allcapsspacing[1]{{\addfontfeature{LetterSpace=15}#1}}
\renewcommand\smallcapsspacing[1]{{\addfontfeature{LetterSpace=10}#1}}

% for url
\usepackage{hyperref}
\hypersetup{colorlinks = true,
  linkcolor = teal,
  urlcolor  = teal,
  citecolor = blue,
  anchorcolor = blue}

\newcommand{\me}[4]{
    \author{
      {\bfseries 姓名:}\underline{#1}\hspace{2em}
      {\bfseries 学号:}\underline{#2}\hspace{2em}\\[10pt]
      {\bfseries 评分:}\underline{#3\hspace{3em}}\hspace{2em}
      {\bfseries 评阅:}\underline{#4\hspace{3em}}
  }
}

% Please ALWAYS Keep This.
\newcommand{\noplagiarism}{
  \begin{center}
    \fbox{\begin{tabular}{@{}c@{}}
      请独立完成作业,不得抄袭。\\
      若得到他人帮助, 请致谢。\\
      若参考了其它资料,请给出引用。\\
      鼓励讨论,但需独立书写解题过程。
    \end{tabular}}
  \end{center}
}

% \newcommand{\goal}[1]{
%   \begin{center}{\fcolorbox{blue}{yellow!60}{\parbox{0.50\textwidth}{\large
%     \begin{itemize}
%       \item 体会``思维的乐趣''
%       \item 初步了解递归与数学归纳法
%       \item 初步接触算法概念与问题下界概念
%     \end{itemize}}}}
%   \end{center}
% }

% Each hw consists of four parts:
\newcommand{\beginrequired}{\hspace{5em}\section{作业 (必做部分)}}
\newcommand{\beginoptional}{\section{作业 (选做部分)}}
\newcommand{\beginot}{\section{Open Topics}}
\newcommand{\begincorrection}{\section{订正}}
\newcommand{\beginfb}{\section{反馈}}

% for math
\usepackage{amsmath, mathtools, amsfonts, amssymb}
\newcommand{\set}[1]{\{#1\}}

% define theorem-like environments
\usepackage[amsmath, thmmarks]{ntheorem}

\theoremstyle{break}
\theorempreskip{2.0\topsep}
\theorembodyfont{\song}
\theoremseparator{}
\newtheorem{problem}{题目}[subsection]
\renewcommand{\theproblem}{\arabic{problem}}
\newtheorem{ot}{Open Topics}

\theorempreskip{3.0\topsep}
\theoremheaderfont{\kai\bfseries}
\theoremseparator{:}
\theorempostwork{\bigskip\hrule}
\newtheorem*{solution}{解答}
\theorempostwork{\bigskip\hrule}
\newtheorem*{revision}{订正}

\theoremstyle{plain}
\newtheorem*{cause}{错因分析}
\newtheorem*{remark}{注}

\theoremstyle{break}
\theorempostwork{\bigskip\hrule}
\theoremsymbol{\ensuremath{\Box}}
\newtheorem*{proof}{证明}

% \newcommand{\ot}{\blue{\bf [OT]}}

% for figs
\renewcommand\figurename{图}
\renewcommand\tablename{表}

% for fig without caption: #1: width/size; #2: fig file
\newcommand{\fig}[2]{
  \begin{figure}[htbp]
    \centering
    \includegraphics[#1]{#2}
  \end{figure}
}
% for fig with caption: #1: width/size; #2: fig file; #3: caption
\newcommand{\figcap}[3]{
  \begin{figure}[htbp]
    \centering
    \includegraphics[#1]{#2}
    \caption{#3}
  \end{figure}
}
% for fig with both caption and label: #1: width/size; #2: fig file; #3: caption; #4: label
\newcommand{\figcaplbl}[4]{
  \begin{figure}[htbp]
    \centering
    \includegraphics[#1]{#2}
    \caption{#3}
    \label{#4}
  \end{figure}
}
% for margin fig without caption: #1: width/size; #2: fig file
\newcommand{\mfig}[2]{
  \begin{marginfigure}
    \centering
    \includegraphics[#1]{#2}
  \end{marginfigure}
}
% for margin fig with caption: #1: width/size; #2: fig file; #3: caption
\newcommand{\mfigcap}[3]{
  \begin{marginfigure}
    \centering
    \includegraphics[#1]{#2}
    \caption{#3}
  \end{marginfigure}
}

\usepackage{fancyvrb}

% for algorithms
\usepackage[]{algorithm}
\usepackage[]{algpseudocode} % noend
% See [Adjust the indentation whithin the algorithmicx-package when a line is broken](https://tex.stackexchange.com/a/68540/23098)
\newcommand{\algparbox}[1]{\parbox[t]{\dimexpr\linewidth-\algorithmicindent}{#1\strut}}
\newcommand{\hStatex}[0]{\vspace{5pt}}
\makeatletter
\newlength{\trianglerightwidth}
\settowidth{\trianglerightwidth}{$\triangleright$~}
\algnewcommand{\LineComment}[1]{\Statex \hskip\ALG@thistlm \(\triangleright\) #1}
\algnewcommand{\LineCommentCont}[1]{\Statex \hskip\ALG@thistlm%
  \parbox[t]{\dimexpr\linewidth-\ALG@thistlm}{\hangindent=\trianglerightwidth \hangafter=1 \strut$\triangleright$ #1\strut}}
\makeatother

% for footnote/marginnote
% see https://tex.stackexchange.com/a/133265/23098
\usepackage{tikz}
\newcommand{\circled}[1]{%
  \tikz[baseline=(char.base)]
  \node [draw, circle, inner sep = 0.5pt, font = \tiny, minimum size = 8pt] (char) {#1};
}
\renewcommand\thefootnote{\protect\circled{\arabic{footnote}}}

\newcommand{\score}[1]{{\bf [#1 分]}}

\newcommand{\rel}[1]{\xrightarrow{#1}}
\newcommand{\dstar}{\xRightarrow[]{\ast}}
\newcommand{\dplus}{\xRightarrow[]{+}}
\newcommand{\lm}{\xRightarrow[\text{lm}]{}}
\renewcommand{\rm}{\xRightarrow[\text{rm}]{}}
\newcommand{\dpluslm}{\xRightarrow[\text{lm}]{+}}
\newcommand{\dstarlm}{\xRightarrow[\text{lm}]{\ast}}
\newcommand{\dplusrm}{\xRightarrow[\text{rm}]{+}}
\newcommand{\dstarrm}{\xRightarrow[\text{rm}]{\ast}}

\newcommand{\sep}{\;\big\lvert\;}

\newcommand{\first}{\textsc{First}}
\newcommand{\follow}{\textsc{Follow}}

% see https://tex.stackexchange.com/a/109906/23098
\usepackage{empheq}
\newcommand*\widefbox[1]{\fbox{\hspace{2em}#1\hspace{2em}}} % feel free to modify this file if you understand LaTeX well
%%%%%%%%%%%%%%%%%%%%
\title{编译原理作业 (2)}
\me{魏恒峰}{hfwei@nju.edu.cn}{}{}
\date{\zhtoday}
%%%%%%%%%%%%%%%%%%%%
\begin{document}
\maketitle
%%%%%%%%%%%%%%%%%%%%
\noplagiarism % PLEASE DON'T DELETE THIS LINE!
%%%%%%%%%%%%%%%%%%%%
\begin{abstract}
  \fig{width = 0.80\textwidth}{figs/automata-quote}
\end{abstract}
%%%%%%%%%%%%%%%%%%%%
\beginrequired

%%%%%%%%%%%%%%%
\begin{problem}[从正则表达式到自动机]
  考虑如下正则表达式:
  \sidenote{
    如何用 \LaTeX{} 写(复杂的)正则表达式?
    \begin{itemize}
      \item \href{https://tex.stackexchange.com/a/162122/23098}{How to escape properly and output regex in latex?@tex.stackexchange}
    \end{itemize}
    如何用 \LaTeX{} 画自动机?
    \begin{itemize}
      \item \href{https://www3.nd.edu/~kogge/courses/cse30151-fa17/Public/other/tikz\_tutorial.pdf}{使用{\texttt{tikz automata}} library}
      \item \href{https://hayesall.com/blog/latex-automata/}{另一个关于\texttt{tikz automata}的教程}
      \item 在 \href{https://notendur.hi.is/aee11/automataLatexGen/}{网站\texttt{automataLatexGen}生成\LaTeX{}代码}
      \item \red{使用 \href{https://www.jflap.org/}{\texttt{jflap}} 工具 ({\bf 推荐学习该工具})}
    \end{itemize}
  }

  \[
    \Big(0|\big(1(01^{\ast}0)^{\ast}1\big)\Big)^{\ast}
  \]

  \begin{enumerate}[(1)]
    \item 请使用 Thompson 构造法将该正则表达式转换为 NFA。
    \item 请使用子集构造法将该 NFA 转换为 DFA。
    \item 请使用 Hopcroft 最小化算法将该 DFA 最小化。
    \item 请解释这个正则表达式为什么表示``3 的倍数(二进制表示)''
    \item 如果有条件, 请让 GPT-4 证明一下。它的证明是正确的吗?
      如果不正确, 有可能在它的证明的基础上进行修正, 得到正确的证明吗?
      或者可以与之交互, 引导它得到正确的证明吗?
      请提交交互记录截图或链接。
    \item (\red{\bf 可选})
      请使用 Kleene 算法 (见同文件夹 Kleene Algorithm PDF 文档) 将最小化的 DFA 转化为正则表达式。
    \item (\red{\bf 可选})
      得到的正则表达式与原正则表达式是否相同?
      如果不同, 是否可以通过一些等价变换得到相同的正则表达式?
  \end{enumerate}
\end{problem}

\begin{solution}
\end{solution}
%%%%%%%%%%%%%%%

%%%%%%%%%%%%%%%
\begin{problem}[设计模式]
  理解 Listener 与 Visitor 设计模式,对于本课程实验至关重要。
  本题可以作为学习笔记,比如从类图、顺序图等角度解释这两种设计模式。
  \begin{enumerate}[(1)]
    \item 请自学 Listener/Observer (监听器) 设计模式~\footnote{参考资料:
      \begin{itemize}
        \item \href{https://github.com/antlr/antlr4/blob/master/doc/listeners.md}{Listener in ANTLR 4}
        \item 《ANTLR 4 权威指南》Section 4.3、Section 7.2
        \item 分析 ANTLR 4 自动生成的语法分析器代码中的 Listener 设计模式
      \end{itemize}
    }。
    \item 请自学 Visitor (访问者) 设计模式~\footnote{参考资料:
      \begin{itemize}
        \item 《ANTLR 4 权威指南》Section 4.2、Section 7.3
        \item \href{https://en.wikipedia.org/wiki/Visitor_pattern}{Vistor Design Pattern @ wiki}
        \item 《Design Patterns Elements of Reusable Object-Oriented Software》 Page 331
        \item 分析 ANTLR 4 自动生成的语法分析器代码中的 Visitor 设计模式
      \end{itemize}
    }。
  \end{enumerate}
  \fig{width = 0.50\textwidth}{figs/design-patterns}
\end{problem}

\begin{solution}
\end{solution}
%%%%%%%%%%%%%%%

\beginoptional
%%%%%%%%%%%%%%%
\begin{problem}[正则表达式练习]
  参考课程录屏中的正则表达式部分或者《正则表达式必知必会》一书,
  使用正则表达式解决以下问题。
  \mfig{width = 0.80\textwidth}{figs/book-learn-re}
  \mfig{width = 1.10\textwidth}{figs/regex-quote}

  \begin{enumerate}[(1)]
    \item Get all `fat' or `mat' words from the input string
      that come after the word `The' or `the'.
      \href{https://regex101.com/r/3x5F4W/1}{(Try it: 可在这里练习)}

      \fig{width = 1.00\textwidth}{figs/fat-mat}
    \item 在 Android 开发中,经常会看到如下所示的错误堆栈信息。
      我们想使用正则表达式提取出错误信息中的方法名、文件名与行号三部分信息。

      \texttt{at package.class.methodname(filename:linenumber)}

      \href{https://regex101.com/r/YeKFVB/1}{(Try it: 可在这里练习)}

      \fig{width = 0.80\textwidth}{figs/android}
  \end{enumerate}
\end{problem}

\begin{solution}
\end{solution}
%%%%%%%%%%%%%%%
%%%%%%%%%%%%%%%%%%%%
\end{document}