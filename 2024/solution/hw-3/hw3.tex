% hw3.tex

% !TEX program = xelatex
%%%%%%%%%%%%%%%%%%%%
% see http://mirrors.concertpass.com/tex-archive/macros/latex/contrib/tufte-latex/sample-handout.pdf
% for how to use tufte-handout
\documentclass[a4paper, justified]{tufte-handout}

% hw-preamble.tex

% geometry for A4 paper
% See https://tex.stackexchange.com/a/119912/23098
\geometry{
  left=20.0mm,
  top=20.0mm,
  bottom=20.0mm,
  textwidth=130mm, % main text block
  marginparsep=5.0mm, % gutter between main text block and margin notes
  marginparwidth=50.0mm % width of margin notes
}

% for colors
\usepackage{xcolor} % usage: \color{red}{text}
% predefined colors
\newcommand{\red}[1]{\textcolor{red}{#1}} % usage: \red{text}
\newcommand{\blue}[1]{\textcolor{blue}{#1}}
\newcommand{\teal}[1]{\textcolor{teal}{#1}}

\usepackage{todonotes}

% heading
\usepackage{sectsty}
\setcounter{secnumdepth}{2}
\allsectionsfont{\centering\huge\rmfamily}

% for Chinese
\usepackage{xeCJK}
\usepackage{zhnumber}
\setCJKmainfont[BoldFont=FandolSong-Bold.otf]{FandolSong-Regular.otf}

% for fonts
\usepackage{fontspec}
\newcommand{\song}{\CJKfamily{song}}
\newcommand{\kai}{\CJKfamily{kai}}

% To fix the ``MakeTextLowerCase'' bug:
% See https://github.com/Tufte-LaTeX/tufte-latex/issues/64#issuecomment-78572017
% Set up the spacing using fontspec features
\renewcommand\allcapsspacing[1]{{\addfontfeature{LetterSpace=15}#1}}
\renewcommand\smallcapsspacing[1]{{\addfontfeature{LetterSpace=10}#1}}

% for url
\usepackage{hyperref}
\hypersetup{colorlinks = true,
  linkcolor = teal,
  urlcolor  = teal,
  citecolor = blue,
  anchorcolor = blue}

\newcommand{\me}[4]{
    \author{
      {\bfseries 姓名:}\underline{#1}\hspace{2em}
      {\bfseries 学号:}\underline{#2}\hspace{2em}\\[10pt]
      {\bfseries 评分:}\underline{#3\hspace{3em}}\hspace{2em}
      {\bfseries 评阅:}\underline{#4\hspace{3em}}
  }
}

% Please ALWAYS Keep This.
\newcommand{\noplagiarism}{
  \begin{center}
    \fbox{\begin{tabular}{@{}c@{}}
      请独立完成作业,不得抄袭。\\
      若得到他人帮助, 请致谢。\\
      若参考了其它资料,请给出引用。\\
      鼓励讨论,但需独立书写解题过程。
    \end{tabular}}
  \end{center}
}

% \newcommand{\goal}[1]{
%   \begin{center}{\fcolorbox{blue}{yellow!60}{\parbox{0.50\textwidth}{\large
%     \begin{itemize}
%       \item 体会``思维的乐趣''
%       \item 初步了解递归与数学归纳法
%       \item 初步接触算法概念与问题下界概念
%     \end{itemize}}}}
%   \end{center}
% }

% Each hw consists of four parts:
\newcommand{\beginrequired}{\hspace{5em}\section{作业 (必做部分)}}
\newcommand{\beginoptional}{\section{作业 (选做部分)}}
\newcommand{\beginot}{\section{Open Topics}}
\newcommand{\begincorrection}{\section{订正}}
\newcommand{\beginfb}{\section{反馈}}

% for math
\usepackage{amsmath, mathtools, amsfonts, amssymb}
\newcommand{\set}[1]{\{#1\}}

% define theorem-like environments
\usepackage[amsmath, thmmarks]{ntheorem}

\theoremstyle{break}
\theorempreskip{2.0\topsep}
\theorembodyfont{\song}
\theoremseparator{}
\newtheorem{problem}{题目}[subsection]
\renewcommand{\theproblem}{\arabic{problem}}
\newtheorem{ot}{Open Topics}

\theorempreskip{3.0\topsep}
\theoremheaderfont{\kai\bfseries}
\theoremseparator{:}
\theorempostwork{\bigskip\hrule}
\newtheorem*{solution}{解答}
\theorempostwork{\bigskip\hrule}
\newtheorem*{revision}{订正}

\theoremstyle{plain}
\newtheorem*{cause}{错因分析}
\newtheorem*{remark}{注}

\theoremstyle{break}
\theorempostwork{\bigskip\hrule}
\theoremsymbol{\ensuremath{\Box}}
\newtheorem*{proof}{证明}

% \newcommand{\ot}{\blue{\bf [OT]}}

% for figs
\renewcommand\figurename{图}
\renewcommand\tablename{表}

% for fig without caption: #1: width/size; #2: fig file
\newcommand{\fig}[2]{
  \begin{figure}[htbp]
    \centering
    \includegraphics[#1]{#2}
  \end{figure}
}
% for fig with caption: #1: width/size; #2: fig file; #3: caption
\newcommand{\figcap}[3]{
  \begin{figure}[htbp]
    \centering
    \includegraphics[#1]{#2}
    \caption{#3}
  \end{figure}
}
% for fig with both caption and label: #1: width/size; #2: fig file; #3: caption; #4: label
\newcommand{\figcaplbl}[4]{
  \begin{figure}[htbp]
    \centering
    \includegraphics[#1]{#2}
    \caption{#3}
    \label{#4}
  \end{figure}
}
% for margin fig without caption: #1: width/size; #2: fig file
\newcommand{\mfig}[2]{
  \begin{marginfigure}
    \centering
    \includegraphics[#1]{#2}
  \end{marginfigure}
}
% for margin fig with caption: #1: width/size; #2: fig file; #3: caption
\newcommand{\mfigcap}[3]{
  \begin{marginfigure}
    \centering
    \includegraphics[#1]{#2}
    \caption{#3}
  \end{marginfigure}
}

\usepackage{fancyvrb}

% for algorithms
\usepackage[]{algorithm}
\usepackage[]{algpseudocode} % noend
% See [Adjust the indentation whithin the algorithmicx-package when a line is broken](https://tex.stackexchange.com/a/68540/23098)
\newcommand{\algparbox}[1]{\parbox[t]{\dimexpr\linewidth-\algorithmicindent}{#1\strut}}
\newcommand{\hStatex}[0]{\vspace{5pt}}
\makeatletter
\newlength{\trianglerightwidth}
\settowidth{\trianglerightwidth}{$\triangleright$~}
\algnewcommand{\LineComment}[1]{\Statex \hskip\ALG@thistlm \(\triangleright\) #1}
\algnewcommand{\LineCommentCont}[1]{\Statex \hskip\ALG@thistlm%
  \parbox[t]{\dimexpr\linewidth-\ALG@thistlm}{\hangindent=\trianglerightwidth \hangafter=1 \strut$\triangleright$ #1\strut}}
\makeatother

% for footnote/marginnote
% see https://tex.stackexchange.com/a/133265/23098
\usepackage{tikz}
\newcommand{\circled}[1]{%
  \tikz[baseline=(char.base)]
  \node [draw, circle, inner sep = 0.5pt, font = \tiny, minimum size = 8pt] (char) {#1};
}
\renewcommand\thefootnote{\protect\circled{\arabic{footnote}}}

\newcommand{\score}[1]{{\bf [#1 分]}}

\newcommand{\rel}[1]{\xrightarrow{#1}}
\newcommand{\dstar}{\xRightarrow[]{\ast}}
\newcommand{\dplus}{\xRightarrow[]{+}}
\newcommand{\lm}{\xRightarrow[\text{lm}]{}}
\renewcommand{\rm}{\xRightarrow[\text{rm}]{}}
\newcommand{\dpluslm}{\xRightarrow[\text{lm}]{+}}
\newcommand{\dstarlm}{\xRightarrow[\text{lm}]{\ast}}
\newcommand{\dplusrm}{\xRightarrow[\text{rm}]{+}}
\newcommand{\dstarrm}{\xRightarrow[\text{rm}]{\ast}}

\newcommand{\forkw}{\text{\bf for}}
\newcommand{\ifkw}{\text{\bf if}}
\newcommand{\printkw}{\text{\bf print}}

\newcommand{\first}{\textsc{First}}
\newcommand{\follow}{\textsc{Follow}}

% see https://tex.stackexchange.com/a/109906/23098
\usepackage{empheq}
\newcommand*\widefbox[1]{\fbox{\hspace{2em}#1\hspace{2em}}} % feel free to modify this file if you understand LaTeX well
%%%%%%%%%%%%%%%%%%%%
\title{编译原理作业 (3)}
\me{\hspace{50pt}}{\hspace{70pt}}
\date{2024年03月31日}
%%%%%%%%%%%%%%%%%%%%
\begin{document}
\maketitle
%%%%%%%%%%%%%%%%%%%%
\noplagiarism % PLEASE DON'T DELETE THIS LINE!
%%%%%%%%%%%%%%%%%%%%
\begin{abstract}
  \fig{width = 0.60\textwidth}{figs/pumping-lemma}
\end{abstract}
%%%%%%%%%%%%%%%%%%%%
\beginrequired
%%%%%%%%%%%%%%%

%%%%%%%%%%%%%%%
\begin{problem}
  请证明以下上下文无关文法表示语言
  $\set{x \in \set{a, b}^{\ast} \mid x \;\text{中}\; a, b \;\text{个数相同}}$。

  % cfg-equal-number-a-b-correct.tex

\begin{align*}
  V \to VV \mid aVb \mid bVa \mid \epsilon
\end{align*}
\end{problem}

\begin{solution}
\begin{enumerate}
    \item[(a)] 首先证明该CFG可表示的任意字符串x都满足: x中a, b个数相同. \\
    由于任何可用该CFG可表示的字符串都对应着至少一颗语法树,我们可以基于语法树高度$h$进行归纳证明. \\
    奠基: \\
    树高$h=1$时, 能够表达的字符串集合$S_0$为$\{\epsilon\}$, 满足每个元素的a,b个数相同. \\
    归纳假设: \\
    树高$h\leq k-1$时, 能够表达的字符串集合$S_h$, 满足每个元素的a,b个数相同. \\
    归纳递推: \\
    树高$h\leq k$时,有三种情况:
    \begin{enumerate}
        \item[1.] 匹配$V \rightarrow V_0V_1$规则,$V_0,V_1$对应的树高都小于k,利用归纳假设可以得到该情况下能够表达的字符串集合$S_k$满足条件. \\
        \item[2.] 匹配$V \rightarrow aV_0b$规则,$V_0$对应的树高小于k,利用归纳假设可以得到该情况下能够表达的字符串集合$S_k$满足条件. \\
        \item[3.] 匹配$V \rightarrow bV_0a$规则,$V_0$对应的树高小于k,利用归纳假设可以得到该情况下能够表达的字符串集合$S_k$满足条件. \\
    \end{enumerate}
    综合三种情况, 此时能够表达的字符串集合$S_k$, 满足每个元素的a,b个数相同.

    \item[(b)] 集合$\{x\in \{a,b\}^* | x$中$a,b$个数相同$\}$中的元素都可以用该CFG表达. \\
    令$L=\set{x \in \set{a, b}^{\ast} \mid x \;\text{中}\; a, b \;\text{个数相同}}$. \\
    对于任意$s\in L$,基于字符串长度$l=|s|$进行归纳. \\
    奠基: \\
    字符串长度$l=0$时, $s$属于的字符串集合$S_0$为$\{\epsilon\}$, 满足每个元素都可以用该CFG表达. \\
    归纳假设:\\ 
    字符串长度$l\leq k-1$时, $s$属于的字符串集合$S_l$, 满足每个元素都可以用该CFG表达. \\
    归纳递推: \\
    字符串长度$\leq k$时,有四种情况
    \begin{enumerate}
        \item[1.] $s$最左端为$a$,最右端为$b$,令$s=as_0b$,因为$|s_0|\leq k-1$于是$s_0$可以用该CFG表达,进而s可以用该CFG表达
        \item[2.] $s$最左端为$b$,最右端为$a$,与情况1.同理
        \item[3.] $s$最左端为$a$,最右端为$a$,令$s=s_0s_1$,且满足$s_0$最左端为$a$,最右端为$b$;$s_1$最左端为$b$,最右端为$a$.(可以很容易证明必然能够找到满足条件的$s_0$和$s_1$)于是$s_0$和$s_1$可以用该CFG表达,进而s可以用该CFG表达.
        \item[4.] $s$最左端为$b$,最右端为$b$,与情况3.同理
    \end{enumerate}
    综合四种情况,此时$s$属于的字符串集合$S_k$, 满足每个元素都可以用该CFG表达. \\
\end{enumerate}
根据(a)(b),可以得到该CFG表示语言$\set{x \in \set{a, b}^{\ast} \mid x \;\text{中}\; a, b \;\text{个数相同}}$
\end{solution}
%%%%%%%%%%%%%%%

\pagebreak
%%%%%%%%%%%%%%%
\begin{problem}
  请证明以下两种 \texttt{if-else} 文法都是无二义性的,
  并且实现了 ``\texttt{else} 与最近的未匹配的 \texttt{if} 匹配'' 的语义。
  \begin{enumerate}[(1)]
    \item 符合 ANTLR 4 中的``最前优先匹配原则''的如下文法:
      \fig{width = 0.80\textwidth}{figs/ifstat-g4}

      尝试: \url{https://github.com/courses-at-nju-by-hfwei/2024-compilers-coding/blob/main/src/main/antlr/ifstat/IfStat.g4}
    \item 教材中改写后的文法:
      \fig{width = 1.00\textwidth}{figs/ifstat-open-matched-g4}

      尝试: \url{https://github.com/courses-at-nju-by-hfwei/2024-compilers-coding/blob/main/src/main/antlr/ifstat/IfStatOpenMatched.g4}
  \end{enumerate}
\end{problem}

\begin{solution}
\begin{enumerate}
    \item[(1)]
    \textbf{无二义性}: \\
    antlr4默认排在前面的语法规则优先级更高,无二义性. \\
    \textbf{满足else与最近的未匹配的if匹配的语义}: \\
    只需证明在\textbf{stat规则二}中的第一个stat不会匹配\textbf{stat规则一}即可,采用反证法: \\
    假设在\textbf{stat规则二}中的第一个stat匹配\textbf{stat规则一},那么根据最前优先匹配规则,else应该与第一个stat中的if匹配,矛盾!\\
    因此\textbf{stat规则二}中的第一个stat不会匹配\textbf{stat规则一},于是满足else与最近的未匹配的if匹配的语义.
    \item[(2)]
    \textbf{无二义性}: \\
    若满足else与最近的未匹配的if匹配的语义,即可说明该文法不无二义性,我们直接证明下一步即可. \\
    \textbf{满足else与最近的未匹配的if匹配的语义}: \\
    与(1)类似,只需证明\textbf{matched\_stat规则一}和\textbf{open\_stat规则二}中的第一个matched\_stat不会匹配\textbf{open\_stat规则一},而显然matched\_stat不能匹配\textbf{open\_stat}的任何规则,因此直接从上下文无关语法层面上保证正确性.
\end{enumerate}
\end{solution}
%%%%%%%%%%%%%%%

\beginoptional

%%%%%%%%%%%%%%%
\begin{problem}
  请使用 Pumping Lemma 证明以下语言不是正则语言:
  \begin{enumerate}[(1)]
    \item $L = \set{x \in \set{a, b}^{\ast} \mid x \;\text{中}\; a, b \;\text{个数相同}}$。
      \footnote{提示: 可以考虑 $s = a^{p} b^{p}$。}
    \item $L = \set{a^{i}b^{j} \mid i > j}$.
      \footnote{提示: 可以考虑 $s = a^{p + 1}b^{p}$。}
  \end{enumerate}
\end{problem}

\begin{solution}
\begin{enumerate}
    \item[(1)] 假设$L$是正则语言,那么令$n$为$L$对应的泵长度. \\
     令$s = a^nb^n \in L$. \\
     根据泵引理,可以令$s=xyz$,其中$|xy|\leq n,|y|\geq 1$. \\
     易知$y$的只能由$a$组成,于是对于$\forall k > 1, xy^kz\notin L$,矛盾! \\
     因此$L$不是正则语言.
     \item[(2)] 假设$L$是正则语言,那么令$n$为$L$对应的泵长度. \\
     令$s = a^{n+1}b^n \in L$. \\
     根据泵引理,可以令$s=xyz$,其中$|xy|\leq n,|y|\geq 1$. \\
     易知$y$的只能由$a$组成,于是对于$\forall k=0, xy^0z=xz\notin L$,矛盾! \\
     因此$L$不是正则语言.
\end{enumerate}
\end{solution}
%%%%%%%%%%%%%%%%%%%%
% 如果没有需要订正的题目,可以把这部分删掉

% \begincorrection
%%%%%%%%%%%%%%%%%%%%

%%%%%%%%%%%%%%%%%%%%
% 如果没有反馈,可以把这部分删掉
\beginfb

请在 Zulip 平台讨论或将反馈发送至 \texttt{hfwei@nju.edu.cn}。
%%%%%%%%%%%%%%%%%%%%
\end{document}