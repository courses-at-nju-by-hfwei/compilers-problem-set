% hw1.tex

% !TEX program = xelatex
%%%%%%%%%%%%%%%%%%%%
% see http://mirrors.concertpass.com/tex-archive/macros/latex/contrib/tufte-latex/sample-handout.pdf
% for how to use tufte-handout
\documentclass[a4paper, justified]{tufte-handout}
\usepackage{listings}
\lstset{language=Java} % 设置代码语言为 Java

% hw-preamble.tex

% geometry for A4 paper
% See https://tex.stackexchange.com/a/119912/23098
\geometry{
  left=20.0mm,
  top=20.0mm,
  bottom=20.0mm,
  textwidth=130mm, % main text block
  marginparsep=5.0mm, % gutter between main text block and margin notes
  marginparwidth=50.0mm % width of margin notes
}

% for colors
\usepackage{xcolor} % usage: \color{red}{text}
% predefined colors
\newcommand{\red}[1]{\textcolor{red}{#1}} % usage: \red{text}
\newcommand{\blue}[1]{\textcolor{blue}{#1}}
\newcommand{\teal}[1]{\textcolor{teal}{#1}}

\usepackage{todonotes}

% heading
\usepackage{sectsty}
\setcounter{secnumdepth}{2}
\allsectionsfont{\centering\huge\rmfamily}

% for Chinese
\usepackage{xeCJK}
\usepackage{zhnumber}
\setCJKmainfont[BoldFont=FandolSong-Bold.otf]{FandolSong-Regular.otf}

% for fonts
\usepackage{fontspec}
\newcommand{\song}{\CJKfamily{song}}
\newcommand{\kai}{\CJKfamily{kai}}

% To fix the ``MakeTextLowerCase'' bug:
% See https://github.com/Tufte-LaTeX/tufte-latex/issues/64#issuecomment-78572017
% Set up the spacing using fontspec features
\renewcommand\allcapsspacing[1]{{\addfontfeature{LetterSpace=15}#1}}
\renewcommand\smallcapsspacing[1]{{\addfontfeature{LetterSpace=10}#1}}

% for url
\usepackage{hyperref}
\hypersetup{colorlinks = true,
  linkcolor = teal,
  urlcolor  = teal,
  citecolor = blue,
  anchorcolor = blue}

\newcommand{\me}[4]{
    \author{
      {\bfseries 姓名:}\underline{#1}\hspace{2em}
      {\bfseries 学号:}\underline{#2}\hspace{2em}\\[10pt]
      {\bfseries 评分:}\underline{#3\hspace{3em}}\hspace{2em}
      {\bfseries 评阅:}\underline{#4\hspace{3em}}
  }
}

% Please ALWAYS Keep This.
\newcommand{\noplagiarism}{
  \begin{center}
    \fbox{\begin{tabular}{@{}c@{}}
      请独立完成作业,不得抄袭。\\
      若得到他人帮助, 请致谢。\\
      若参考了其它资料,请给出引用。\\
      鼓励讨论,但需独立书写解题过程。
    \end{tabular}}
  \end{center}
}

% \newcommand{\goal}[1]{
%   \begin{center}{\fcolorbox{blue}{yellow!60}{\parbox{0.50\textwidth}{\large
%     \begin{itemize}
%       \item 体会``思维的乐趣''
%       \item 初步了解递归与数学归纳法
%       \item 初步接触算法概念与问题下界概念
%     \end{itemize}}}}
%   \end{center}
% }

% Each hw consists of four parts:
\newcommand{\beginrequired}{\hspace{5em}\section{作业 (必做部分)}}
\newcommand{\beginoptional}{\section{作业 (选做部分)}}
\newcommand{\beginot}{\section{Open Topics}}
\newcommand{\begincorrection}{\section{订正}}
\newcommand{\beginfb}{\section{反馈}}

% for math
\usepackage{amsmath, mathtools, amsfonts, amssymb}
\newcommand{\set}[1]{\{#1\}}

% define theorem-like environments
\usepackage[amsmath, thmmarks]{ntheorem}

\theoremstyle{break}
\theorempreskip{2.0\topsep}
\theorembodyfont{\song}
\theoremseparator{}
\newtheorem{problem}{题目}[subsection]
\renewcommand{\theproblem}{\arabic{problem}}
\newtheorem{ot}{Open Topics}

\theorempreskip{3.0\topsep}
\theoremheaderfont{\kai\bfseries}
\theoremseparator{:}
\theorempostwork{\bigskip\hrule}
\newtheorem*{solution}{解答}
\theorempostwork{\bigskip\hrule}
\newtheorem*{revision}{订正}

\theoremstyle{plain}
\newtheorem*{cause}{错因分析}
\newtheorem*{remark}{注}

\theoremstyle{break}
\theorempostwork{\bigskip\hrule}
\theoremsymbol{\ensuremath{\Box}}
\newtheorem*{proof}{证明}

% \newcommand{\ot}{\blue{\bf [OT]}}

% for figs
\renewcommand\figurename{图}
\renewcommand\tablename{表}

% for fig without caption: #1: width/size; #2: fig file
\newcommand{\fig}[2]{
  \begin{figure}[htbp]
    \centering
    \includegraphics[#1]{#2}
  \end{figure}
}
% for fig with caption: #1: width/size; #2: fig file; #3: caption
\newcommand{\figcap}[3]{
  \begin{figure}[htbp]
    \centering
    \includegraphics[#1]{#2}
    \caption{#3}
  \end{figure}
}
% for fig with both caption and label: #1: width/size; #2: fig file; #3: caption; #4: label
\newcommand{\figcaplbl}[4]{
  \begin{figure}[htbp]
    \centering
    \includegraphics[#1]{#2}
    \caption{#3}
    \label{#4}
  \end{figure}
}
% for margin fig without caption: #1: width/size; #2: fig file
\newcommand{\mfig}[2]{
  \begin{marginfigure}
    \centering
    \includegraphics[#1]{#2}
  \end{marginfigure}
}
% for margin fig with caption: #1: width/size; #2: fig file; #3: caption
\newcommand{\mfigcap}[3]{
  \begin{marginfigure}
    \centering
    \includegraphics[#1]{#2}
    \caption{#3}
  \end{marginfigure}
}

\usepackage{fancyvrb}

% for algorithms
\usepackage[]{algorithm}
\usepackage[]{algpseudocode} % noend
% See [Adjust the indentation whithin the algorithmicx-package when a line is broken](https://tex.stackexchange.com/a/68540/23098)
\newcommand{\algparbox}[1]{\parbox[t]{\dimexpr\linewidth-\algorithmicindent}{#1\strut}}
\newcommand{\hStatex}[0]{\vspace{5pt}}
\makeatletter
\newlength{\trianglerightwidth}
\settowidth{\trianglerightwidth}{$\triangleright$~}
\algnewcommand{\LineComment}[1]{\Statex \hskip\ALG@thistlm \(\triangleright\) #1}
\algnewcommand{\LineCommentCont}[1]{\Statex \hskip\ALG@thistlm%
  \parbox[t]{\dimexpr\linewidth-\ALG@thistlm}{\hangindent=\trianglerightwidth \hangafter=1 \strut$\triangleright$ #1\strut}}
\makeatother

% for footnote/marginnote
% see https://tex.stackexchange.com/a/133265/23098
\usepackage{tikz}
\newcommand{\circled}[1]{%
  \tikz[baseline=(char.base)]
  \node [draw, circle, inner sep = 0.5pt, font = \tiny, minimum size = 8pt] (char) {#1};
}
\renewcommand\thefootnote{\protect\circled{\arabic{footnote}}}

\newcommand{\score}[1]{{\bf [#1 分]}}

\newcommand{\rel}[1]{\xrightarrow{#1}}
\newcommand{\dstar}{\xRightarrow[]{\ast}}
\newcommand{\dplus}{\xRightarrow[]{+}}
\newcommand{\lm}{\xRightarrow[\text{lm}]{}}
\renewcommand{\rm}{\xRightarrow[\text{rm}]{}}
\newcommand{\dpluslm}{\xRightarrow[\text{lm}]{+}}
\newcommand{\dstarlm}{\xRightarrow[\text{lm}]{\ast}}
\newcommand{\dplusrm}{\xRightarrow[\text{rm}]{+}}
\newcommand{\dstarrm}{\xRightarrow[\text{rm}]{\ast}}

\newcommand{\sep}{\;\big\lvert\;}

\newcommand{\first}{\textsc{First}}
\newcommand{\follow}{\textsc{Follow}}

% see https://tex.stackexchange.com/a/109906/23098
\usepackage{empheq}
\newcommand*\widefbox[1]{\fbox{\hspace{2em}#1\hspace{2em}}} % feel free to modify this file if you understand LaTeX well
%%%%%%%%%%%%%%%%%%%%
\title{编译原理作业 (1)}
\me{魏恒峰}{hfwei@nju.edu.cn}{}{}
\date{\zhtoday}
%%%%%%%%%%%%%%%%%%%%
\begin{document}
\maketitle
%%%%%%%%%%%%%%%%%%%%
\noplagiarism % PLEASE DON'T DELETE THIS LINE!
%%%%%%%%%%%%%%%%%%%%
\begin{abstract}
  \fig{width = 0.30\textwidth}{figs/lexing}
\end{abstract}
%%%%%%%%%%%%%%%%%%%%
\beginrequired
%%%%%%%%%%%%%%%
\begin{problem}[C 语言中的 ANTLR 4 词法规约]
  阅读 \href{https://github.com/antlr/grammars-v4/blob/master/c/C.g4}{C 语言 \texttt{.g4}
  文件}, 完成以下任务 (均可用图示辅助解释)。
  \begin{enumerate}[(1)]
    \item 定位并解释其中的``字符串'' (\texttt{StringLiteral}) 词法规则。
    \item 定义并解释其中的``常量'' (\texttt{IntegerConstant}、\texttt{FloatingConstant}、\texttt{CharacterConstant}) 词法规则。
    \item 使用 \texttt{lexer grammar}~\footnote{
      关于 \texttt{lexer grammar} 的用法:
      \begin{itemize}
        \item 见《ANTLR 4 权威指南》第 4.1 节
        \item 注意: Gradle ANTLR 插件在需要将 lexer grammar 导入到更大的 grammar 文件中时
          有一个尚未修复的 ``幺蛾子'' (bug), 参见 \href{https://github.com/courses-at-nju-by-hfwei/2024-compilers-coding-0/blob/5bac437778703698372b4913d0d5197e7890bc02/build.gradle\#L34}{build.gradle 文件}。
      \end{itemize}} 在 ANTLR 4 工具中测试 \texttt{C.g4} 中的词法单元
      \begin{itemize}
        \item 考虑如何设计测试用例覆盖尽可能多的情况?
        \item 检查你对词法规则的理解是否与 ANTLR 4 的输出一致。
      \end{itemize}
    \item 其它: 请自行挖掘有趣的内容。
  \end{enumerate}
\end{problem}

\begin{solution}
  \begin{enumerate}[(1)]
    \item ``字符串''(\texttt{StringLiteral})词法规则由由四个部分组成,分别是可选的编码前缀\texttt{EncodingPrefix}、双引号、\texttt{SCharSequence}和另一个双引号。整体上,它表示了一个字符串常量的结构。接下来,我们将依次解释\texttt{EncodingPrefix}和\texttt{SCharSequence}这两个部分。
      \begin{itemize}
        \item \texttt{EncodingPrefix}规则定义了字符串的编码前缀。它可以是`u8'、`u'、`U'或`L'之一。这些前缀用于指示字符串的字符编码方式,比如UTF-8 (u8)、UTF-16 (u和U)或宽字符(L)。
        \item \texttt{SCharSequence}规则定义了定义了\texttt{SChar}的序列,即字符串中的字符序列。它由一个或多个\texttt{SChar}组成。而\texttt{SChar}定义了字符串中的单个字符。它可以是除了双引号、反斜杠和换行符之外的任何字符。如果字符是反斜杠,则可能是转义序列\texttt{EscapeSequence}。此外,该规则还允许处理转义序列$\backslash$n(表示换行)和$\backslash$r$\backslash$n(表示回车换行)。接下来,我们将解释\texttt{EscapeSequence}。\texttt{EscapeSequence}定义了转义序列的形式。转义序列允许在字符串中插入一些特殊字符,如换行符、制表符等。
它可以是简单转义序列\texttt{SimpleEscapeSequence}、八进制转义序列\texttt{OctalEscapeSequence}、十六进制转义序列\texttt{HexadecimalEscapeSequence}或通用字符名称\texttt{UniversalCharacterName}之一:
          \begin{itemize}
            \item \texttt{SimpleEscapeSequence}定义了简单转义序列的形式,比如$\backslash$n(换行符)、$\backslash$t(制表符)等。
            \item \texttt{OctalEscapeSequence}定义了八进制转义序列的形式,比如$\backslash$0、$\backslash$123等。
            \item \texttt{HexadecimalEscapeSequence}定义了十六进制转义序列的形式,比如$\backslash$x1F等。
          \end{itemize}
        \item \texttt{EscapeSequence}在`$\backslash$$\backslash$$\backslash$n'前面可能是``$\backslash$$\backslash$$\backslash$n' (也包括下一条规则 '$\backslash$$\backslash$$\backslash$r$\backslash$n')是为了处理 C 语言中的多行字符串字面量与多行宏中的换行的。
      \end{itemize}
    \item
      \begin{itemize}
        \item 整数常量\texttt{IntegerConstant}由四个部分组成,分别是十进制常量(\texttt{DecimalConstant})、八进制常量(\texttt{OctalConstant})、十六进制常量(\texttt{HexadecimalConstant})和二进制常量(\texttt{BinaryConstant})。除了二进制常量外,每个部分都可以跟随一个整数后缀(\texttt{IntegerSuffix})。
          \begin{itemize}
            \item \texttt{BinaryConstant}:以`0b'或0B为前缀,后面跟着一个由0和1组成的二进制数字序列。
            \item \texttt{DecimalConstant}:以1到9之间的非零数字开头,后面可以跟着任意数量的数字(0-9)。
            \item \texttt{OctalConstant}:以数字0开头,后面跟着零个或多个八进制数字(0-7)。
            \item \texttt{HexadecimalConstant}:以`0x'或`0X'为前缀,后面跟着至少一个十六进制数字(0-9,a-f,A-F)。
            \item \texttt{IntegerSuffix}:整数后缀由可选的无符号后缀(\texttt{UnsignedSuffix})和可选的长整数后缀(\texttt{LongSuffix或LongLongSuffix})组成。无符号后缀可以是`u'或`U'。长整数后缀可以是`l'、`L'、`ll'或`LL',分别表示长整数和长长整数。。
          \end{itemize}
        \item 浮点数常量\texttt{FloatingConstant}可以是十进制浮点常量(\texttt{DecimalFloatingConstant})或十六进制浮点常量(\texttt{HexadecimalFloatingConstant})之一。
          \begin{itemize}
            \item 十进制浮点常量\texttt{DecimalFloatingConstant}由两个部分组成:小数部分(\texttt{FractionalConstant})、指数部分(\texttt{ExponentPart})和浮点数后缀(\texttt{FloatingSuffix})以及数字序列(\texttt{DigitSequence})、指数部分(\texttt{ExponentPart})和浮点数后缀(\texttt{FloatingSuffix})。小数部分可以是小数点前的数字序列,后跟小数点,再跟小数点后的数字序列;或者是仅有小数点后的数字序列。指数部分以`e'或`E'开头,可以带正负号,后面跟着一个数字序列。
            \item 十六进制浮点常量\texttt{HexadecimalFloatingConstant} 由四个部分组成:十六进制前缀(\texttt{HexadecimalPrefix})、十六进制小数部分(\texttt{HexadecimalFractionalConstant}或\texttt{HexadecimalDigitSequence})、二进制指数部分(\texttt{BinaryExponentPart})和浮点数后缀(\texttt{FloatingSuffix})。十六进制前缀以`0x'或`0X'开头。十六进制小数部分可以是可选的十六进制数字序列,后跟小数点,再跟着另一个十六进制数字序列;或者仅有十六进制数字序列。二进制指数部分以`p'或`P'开头,可以带正负号,后面跟着一个数字序列。
            \item \texttt{FloatingSuffix}可以是`f'、`F'、`l'或`L',表示浮点数的类型(单精度浮点数、双精度浮点数、长浮点数等)。
          \end{itemize}
        \item \texttt{CharacterConstant}字符常量由一个单引号、\texttt{CCharSequence}和另一个单引号组成。其中\texttt{CCharSequence}是一个或多个\texttt{CChar}组成的序列。字符常量可以具有不同的前缀:`L'、`u'或`U',用于表示不同的字符类型。\texttt{CChar}定义了字符常量中的单个字符。它可以是除了单引号、反斜杠和换行符之外的任何字符。如果字符是反斜杠,则可能是转义序列(\texttt{EscapeSequence})。\texttt{EscapeSequence}定义了转义序列的形式。转义序列允许在字符常量中插入一些特殊字符,如换行符、制表符等。它也可以是简单转义序列(\texttt{SimpleEscapeSequence})、八进制转义序列(\texttt{OctalEscapeSequence})、十六进制转义序列(\texttt{HexadecimalEscapeSequence})或通用字符名称(\texttt{UniversalCharacterName})之一。这里只介绍通用字符名称(\texttt{UniversalCharacterName})。
          \begin{itemize}
            \item \texttt{UniversalCharacterName}定义了通用字符名称的形式,以$\backslash$u或$\backslash$U开头,后跟四个或八个十六进制数字。\texttt{HexQuad}定义了四个连续的十六进制数字,用于表示通用字符名称中的码点。
          \end{itemize}
      \end{itemize}
    \item 为了尽可能覆盖C语言的词法单元,可以考虑设计以下几种情况:
      \begin{itemize}
        \item 测试基本词法单元:添加各种基本数据类型的常量,如整数常量、浮点数常量、字符常量和各种进制的常量;添加标识符和关键字。
        \item 测试边界情况:可以设计整数和浮点数常量的最小值、最大值以及边界情况;设计包含特殊字符的标识符。
        \item 测试转义序列和通用字符:识别输出各种转义字符;识别通用字符。
        \item 测试不合法字符识别:添加错误的标识符或者未定义的字符;添加错误的转义序列和通用字符。
      \end{itemize}
  \end{enumerate}
\end{solution}
%%%%%%%%%%%%%%%

%%%%%%%%%%%%%%%
\begin{problem}[词法分析器代码分析]
  查看 \href{https://clang.llvm.org/doxygen/classclang_1_1Lexer.html}{Clang Lexer 文档},
  阅读 \href{https://clang.llvm.org/doxygen/Lexer_8cpp_source.html}{Clang 词法分析器源码 \texttt{Lexer.cpp}},
  完成以下任务 (均可用图示辅助解释)。
  \begin{itemize}
    \item 整理函数 \href{https://clang.llvm.org/doxygen/Lexer_8cpp_source.html\#l03669}{\texttt{Lexer::Lex()}}
      的主要逻辑。
    \item 定义到处理 \texttt{StringLiteral} 词法单元的代码,
      并分析代码的主要逻辑。
    \item 定义到处理 \texttt{IntegerConstant} 与 \texttt{FloatingConstant} 词法单元的代码,
      并分析代码的主要逻辑。
    \item 其它: 请自行挖掘有趣的内容。
  \end{itemize}
\end{problem}

\begin{solution}
  \begin{itemize}
    \item \texttt{Lexer::Lex()}函数主要逻辑如下:
      \begin{itemize}
        \item assert(!isDependencyDirectivesLexer());:这是一个断言语句,用于确保词法分析器不处于依赖指令的词法分析模式下。如果该条件为真,将导致断言失败,表示出现了意料之外的情况。
        \item Result.startToken();:开始一个新的标记。
        \item 设置各种标记的标志,这些标志指示标记的一些属性,如是否在行的开头(Token::StartOfLine)、是否具有前导空格(Token::LeadingSpace)、是否具有前导空的宏(Token::LeadingEmptyMacro)等。
        \item `bool atPhysicalStartOfLine = IsAtPhysicalStartOfLine;`将变量IsAtPhysicalStartOfLine的当前值复制给atPhysicalStartOfLine,用于记录是否在物理行的起始位置。这是为了在调用LexTokenInternal函数之前保留IsAtPhysicalStartOfLine的值,以便在需要时了解当前位置是否在物理行的起始位置。
        \item `IsAtPhysicalStartOfLine = false;'将IsAtPhysicalStartOfLine的值设置为false,表示当前不在物理行的起始位置。这是因为在开始处理新的标记之前,词法分析器不再处于物理行的起始位置。
        \item `bool isRawLex = isLexingRawMode();'调用isLexingRawMode()函数,用于检查词法分析器是否处于原始词法分析模式(即以原始模式进行词法分析,不执行任何预处理)。返回值存储在isRawLex变量中。
        \item `(void) isRawLex;'这是一个类型转换语句,将isRawLex变量的值强制转换为void类型,从而避免编译器产生“未使用变量”的警告。在这里,它的目的是告诉编译器,我们有意不使用isRawLex变量,但是调用了isLexingRawMode()函数来检查词法分析器是否处于原始词法分析模式。
        \item 接下来调用LexTokenInternal函数,这是词法分析器内部用于实际识别和解析标记的函数。它将填充Result中的标记对象,并返回一个布尔值,指示是否成功识别了标记。
        \item returnedToken变量接收LexTokenInternal函数的返回值,用于检查是否成功识别了标记。
        \item 最后,如果词法分析处于原始词法分析模式(isRawLex为真),则断言确保词法分析成功。否则,返回returnedToken,指示词法分析是否成功。
      \end{itemize}
    \item StringLiteral词法单元处理函数主要有两个LexStringLiteral和LexRawStringLiteral。
      \begin{itemize}
        \item `LexStringLiteral'函数处理处理的是普通的字符串常量,逻辑如下:
          \begin{itemize}
            \item 扫描字符串常量:从指定的字符指针 CurPtr 开始扫描源代码字符流,直到遇到字符串常量的结束引号(")。
            \item 处理转义字符:在扫描过程中,处理转义字符,包括反斜杠后的转义序列,如 $\backslash$n、$\backslash$t 等。转义字符会被逐个处理,以确保正确解析字符串内容。
            \item 处理特殊情况:在处理过程中,如果遇到换行符($\backslash$n、$\backslash$r)、文件结束符(EOF),或者文件未正确结束,则发出相应的诊断信息,并生成一个未知的标记。
            \item 检查是否存在空字符:如果字符串中存在空字符('$\backslash$0'),则发出警告。
            \item 识别可选的用户定义后缀(UD-suffix):如果在 C++11 模式下,识别并处理可选的用户定义后缀。
            \item 构建标记:构建识别到的字符串常量为标记,并设置其类型为 tok::string\_literal。设置标记的文本范围,并更新 BufferPtr 的位置。
          \end{itemize}
        \item LexRawStringLiteral函数处理逻辑如下:
          \begin{itemize}
            \item 扫描原始字符串常量:从指定的字符指针 CurPtr 开始扫描源代码字符流,直到找到与原始字符串的结束定界符匹配的位置。
            \item 检查原始字符串定界符:检查原始字符串的定界符是否合法,即是否以 `R' 或 `LR' 或 `u8R' 或 `uR' 或 `UR' 开头,后接 (。如果不合法,则发出相应的诊断信息,并尝试寻找下一个 (")。
            \item 识别可选的用户定义后缀(UD-suffix):如果在 C++11 模式下,识别并处理可选的用户定义后缀。
            \item 构建标记:构建识别到的原始字符串常量为标记,并设置其类型为 tok::string\_literal。设置标记的文本范围,并更新 BufferPtr 的位置。
          \end{itemize}
      \end{itemize}
    \item IntegerConstant和FloatConstant词法单元处理的函数是LexNumericConstant。逻辑如下:
      \begin{itemize}
        \item 扫描数字常量:从指定的字符指针CurPtr开始扫描源代码字符流,识别整数或浮点数常量的字符序列。
        \item 识别数字常量:通过`isPreprocessingNumberBody'函数判断当前字符是否属于整数或浮点数的字符序列,循环扫描直到不再是数字常量的一部分。在扫描过程中,通过`ConsumeChar'函数逐步移动字符指针,并存储识别到的字符。
        \item 处理特殊情况:对于特殊情况,如指数表示的浮点数常量(如1e+12)、十六进制浮点数常量以及数字分隔符(C++14/C23新增),进行额外处理;对于指数表示的浮点数常量,检查是否后续还有符号,并在Microsoft模式下特殊处理;对于十六进制浮点数常量,判断是否处于不支持的模式下,例如不支持的C99模式,或者存在下划线的C++17模式;对于数字分隔符,检查其是否符合C++14或C23的要求,如果符合则发出警告。
        \item 处理转移序列和UTF-8字符:对于可能存在于数字常量中的转义序列和UTF-8字符进行处理,包括Unicode字符名称(UCN)和UTF-8字符。
        \item 构建标记:将识别到的数字常量构建为标记(Token),设置其类型为`tok::numeric\_constant'。设置标记的文本范围,并更新CurPtr和BufferPtr的位置。
        \item 返回结果:返回true表示成功识别数字常量,并存储到传入的Token对象中。
      \end{itemize}
  \end{itemize}
\end{solution}
%%%%%%%%%%%%%%%

\beginoptional
%%%%%%%%%%%%%%%
\begin{problem}[手写词法分析器]
  \begin{itemize}
    \item 为 \texttt{FloatingConstant} 词法单元手写词法分析器,
      通过与 ANTLR 4 的输出进行对比检查正确性。
      建议画出状态转移图。
  \end{itemize}
\end{problem}

\begin{solution}
  下面给出一个简单的Java实现的浮点数识别的代码:
  \begin{lstlisting}
    public class Main {
    public static String lexFloatingConstant(String inputString) {
        int position = 0;
        char currentChar = inputString.charAt(position);

        StringBuilder floatingConstant = new StringBuilder();

        // Initial state
        String state = "start";

        while (currentChar != '\0') {
            if (state.equals("start")) {
                if (Character.isDigit(currentChar)) {
                    floatingConstant.append(currentChar);
                    state = "digit_sequence";
                } else if (currentChar == '.') {
                    floatingConstant.append(currentChar);
                    state = "fractional_constant";
                } else {
                    break;
                }
            } else if (state.equals("digit_sequence")) {
                if (Character.isDigit(currentChar)) {
                    floatingConstant.append(currentChar);
                } else if (currentChar == 'e' || currentChar == 'E' || currentChar == 'p' || currentChar == 'P') {
                    floatingConstant.append(currentChar);
                    state = "exponent_part";
                } else if (currentChar == '.') {
                    floatingConstant.append(currentChar);
                    state = "fractional_part";
                } else {
                    break;
                }
            } else if (state.equals("fractional_constant")) {
                if (Character.isDigit(currentChar)) {
                    floatingConstant.append(currentChar);
                    state = "fractional_part";
                } else {
                    break;
                }
            } else if (state.equals("fractional_part")) {
                if (Character.isDigit(currentChar)) {
                    floatingConstant.append(currentChar);
                } else if (currentChar == 'e' || currentChar == 'E' || currentChar == 'p' || currentChar == 'P') {
                    floatingConstant.append(currentChar);
                    state = "exponent_part";
                } else {
                    break;
                }
            } else if (state.equals("exponent_part")) {
                if (currentChar == '+' || currentChar == '-') {
                    floatingConstant.append(currentChar);
                    state = "exponent_sign";
                } else if (Character.isDigit(currentChar)) {
                    floatingConstant.append(currentChar);
                    state = "exponent_digit_sequence";
                } else {
                    break;
                }
            } else if (state.equals("exponent_sign")) {
                if (Character.isDigit(currentChar)) {
                    floatingConstant.append(currentChar);
                    state = "exponent_digit_sequence";
                } else {
                    break;
                }
            } else if (state.equals("exponent_digit_sequence")) {
                if (Character.isDigit(currentChar)) {
                    floatingConstant.append(currentChar);
                } else {
                    break;
                }
            }

            position++;
            if (position < inputString.length()) {
                currentChar = inputString.charAt(position);
            } else {
                currentChar = '\0';
            }
        }

        return floatingConstant.toString().trim();
    }

    public static void main(String[] args) {
        String inputString = "3.14E+12";
        String floatingConstant = lexFloatingConstant(inputString);
        System.out.println(floatingConstant);
    }
}
  \end{lstlisting}
  状态转移图的文字描述如下,首先给出状态含义:
  \begin{itemize}
    \item start:初始状态,开始识别浮点数常量。
    \item digit\_sequence:识别数字序列部分。
    \item fractional\_constant:识别小数部分。
    \item fractional\_part:继续识别小数部分。
    \item exponent\_part:识别指数部分。
    \item exponent\_sign:识别指数部分的正负号。
    \item exponent\_digit\_sequence:识别指数部分的数字序列。
  \end{itemize}
  接下来,我们给出状态转移描述:
  \begin{enumerate}[(1)]
    \item 从初始状态 `start' 开始。
    \item 如果当前字符是数字([0-9]),转移到状态 "digit\_sequence"。
    \item 如果当前字符是小数点('.'),转移到状态 "fractional\_constant"。
    \item 如果当前字符是 'e' 或 'E' 或 'p' 或 'P',转移到状态 "exponent\_part"。
    \item 如果当前状态是 "digit\_sequence",可能转移到状态 "digit\_sequence"、"fractional\_part" 或 "exponent\_part",取决于下一个字符。
    \item 如果当前状态是 "fractional\_constant",可能转移到状态 "fractional\_part",取决于下一个字符。
    \item 如果当前状态是 "fractional\_part",可能继续识别数字序列,也可能转移到状态 "exponent\_part",取决于下一个字符。
    \item 如果当前状态是 "exponent\_part",可能转移到状态 "exponent\_sign",取决于下一个字符。
    \item 如果当前状态是 "exponent\_sign",可能转移到状态 "exponent\_digit\_sequence",取决于下一个字符。
    \item 如果当前状态是 "exponent\_digit\_sequence",可能继续识别数字序列。
  \end{enumerate}
\end{solution}
%%%%%%%%%%%%%%%

%%%%%%%%%%%%%%%%%%%%
% 如果没有需要订正的题目,可以把这部分删掉

% \begincorrection
%%%%%%%%%%%%%%%%%%%%

%%%%%%%%%%%%%%%%%%%%
% 如果没有反馈,可以把这部分删掉
\beginfb

请在 Zulip 上讨论对作业或者课程的意见。
%%%%%%%%%%%%%%%%%%%%
\end{document}