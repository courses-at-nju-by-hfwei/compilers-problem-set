% hw5.tex

% !TEX program = xelatex
%%%%%%%%%%%%%%%%%%%%
% see http://mirrors.concertpass.com/tex-archive/macros/latex/contrib/tufte-latex/sample-handout.pdf
% for how to use tufte-handout
\documentclass[a4paper, justified]{tufte-handout}
\usepackage{listings}
\usepackage{color}
\usepackage{xcolor}

% hw-preamble.tex

% geometry for A4 paper
% See https://tex.stackexchange.com/a/119912/23098
\geometry{
  left=20.0mm,
  top=20.0mm,
  bottom=20.0mm,
  textwidth=130mm, % main text block
  marginparsep=5.0mm, % gutter between main text block and margin notes
  marginparwidth=50.0mm % width of margin notes
}

% for colors
\usepackage{xcolor} % usage: \color{red}{text}
% predefined colors
\newcommand{\red}[1]{\textcolor{red}{#1}} % usage: \red{text}
\newcommand{\blue}[1]{\textcolor{blue}{#1}}
\newcommand{\teal}[1]{\textcolor{teal}{#1}}

\usepackage{todonotes}

% heading
\usepackage{sectsty}
\setcounter{secnumdepth}{2}
\allsectionsfont{\centering\huge\rmfamily}

% for Chinese
\usepackage{xeCJK}
\usepackage{zhnumber}
\setCJKmainfont[BoldFont=FandolSong-Bold.otf]{FandolSong-Regular.otf}

% for fonts
\usepackage{fontspec}
\newcommand{\song}{\CJKfamily{song}}
\newcommand{\kai}{\CJKfamily{kai}}

% To fix the ``MakeTextLowerCase'' bug:
% See https://github.com/Tufte-LaTeX/tufte-latex/issues/64#issuecomment-78572017
% Set up the spacing using fontspec features
\renewcommand\allcapsspacing[1]{{\addfontfeature{LetterSpace=15}#1}}
\renewcommand\smallcapsspacing[1]{{\addfontfeature{LetterSpace=10}#1}}

% for url
\usepackage{hyperref}
\hypersetup{colorlinks = true,
  linkcolor = teal,
  urlcolor  = teal,
  citecolor = blue,
  anchorcolor = blue}

\newcommand{\me}[4]{
    \author{
      {\bfseries 姓名:}\underline{#1}\hspace{2em}
      {\bfseries 学号:}\underline{#2}\hspace{2em}\\[10pt]
      {\bfseries 评分:}\underline{#3\hspace{3em}}\hspace{2em}
      {\bfseries 评阅:}\underline{#4\hspace{3em}}
  }
}

% Please ALWAYS Keep This.
\newcommand{\noplagiarism}{
  \begin{center}
    \fbox{\begin{tabular}{@{}c@{}}
      请独立完成作业,不得抄袭。\\
      若得到他人帮助, 请致谢。\\
      若参考了其它资料,请给出引用。\\
      鼓励讨论,但需独立书写解题过程。
    \end{tabular}}
  \end{center}
}

% \newcommand{\goal}[1]{
%   \begin{center}{\fcolorbox{blue}{yellow!60}{\parbox{0.50\textwidth}{\large
%     \begin{itemize}
%       \item 体会``思维的乐趣''
%       \item 初步了解递归与数学归纳法
%       \item 初步接触算法概念与问题下界概念
%     \end{itemize}}}}
%   \end{center}
% }

% Each hw consists of four parts:
\newcommand{\beginrequired}{\hspace{5em}\section{作业 (必做部分)}}
\newcommand{\beginoptional}{\section{作业 (选做部分)}}
\newcommand{\beginot}{\section{Open Topics}}
\newcommand{\begincorrection}{\section{订正}}
\newcommand{\beginfb}{\section{反馈}}

% for math
\usepackage{amsmath, mathtools, amsfonts, amssymb}
\newcommand{\set}[1]{\{#1\}}

% define theorem-like environments
\usepackage[amsmath, thmmarks]{ntheorem}

\theoremstyle{break}
\theorempreskip{2.0\topsep}
\theorembodyfont{\song}
\theoremseparator{}
\newtheorem{problem}{题目}[subsection]
\renewcommand{\theproblem}{\arabic{problem}}
\newtheorem{ot}{Open Topics}

\theorempreskip{3.0\topsep}
\theoremheaderfont{\kai\bfseries}
\theoremseparator{:}
\theorempostwork{\bigskip\hrule}
\newtheorem*{solution}{解答}
\theorempostwork{\bigskip\hrule}
\newtheorem*{revision}{订正}

\theoremstyle{plain}
\newtheorem*{cause}{错因分析}
\newtheorem*{remark}{注}

\theoremstyle{break}
\theorempostwork{\bigskip\hrule}
\theoremsymbol{\ensuremath{\Box}}
\newtheorem*{proof}{证明}

% \newcommand{\ot}{\blue{\bf [OT]}}

% for figs
\renewcommand\figurename{图}
\renewcommand\tablename{表}

% for fig without caption: #1: width/size; #2: fig file
\newcommand{\fig}[2]{
  \begin{figure}[htbp]
    \centering
    \includegraphics[#1]{#2}
  \end{figure}
}
% for fig with caption: #1: width/size; #2: fig file; #3: caption
\newcommand{\figcap}[3]{
  \begin{figure}[htbp]
    \centering
    \includegraphics[#1]{#2}
    \caption{#3}
  \end{figure}
}
% for fig with both caption and label: #1: width/size; #2: fig file; #3: caption; #4: label
\newcommand{\figcaplbl}[4]{
  \begin{figure}[htbp]
    \centering
    \includegraphics[#1]{#2}
    \caption{#3}
    \label{#4}
  \end{figure}
}
% for margin fig without caption: #1: width/size; #2: fig file
\newcommand{\mfig}[2]{
  \begin{marginfigure}
    \centering
    \includegraphics[#1]{#2}
  \end{marginfigure}
}
% for margin fig with caption: #1: width/size; #2: fig file; #3: caption
\newcommand{\mfigcap}[3]{
  \begin{marginfigure}
    \centering
    \includegraphics[#1]{#2}
    \caption{#3}
  \end{marginfigure}
}

\usepackage{fancyvrb}

% for algorithms
\usepackage[]{algorithm}
\usepackage[]{algpseudocode} % noend
% See [Adjust the indentation whithin the algorithmicx-package when a line is broken](https://tex.stackexchange.com/a/68540/23098)
\newcommand{\algparbox}[1]{\parbox[t]{\dimexpr\linewidth-\algorithmicindent}{#1\strut}}
\newcommand{\hStatex}[0]{\vspace{5pt}}
\makeatletter
\newlength{\trianglerightwidth}
\settowidth{\trianglerightwidth}{$\triangleright$~}
\algnewcommand{\LineComment}[1]{\Statex \hskip\ALG@thistlm \(\triangleright\) #1}
\algnewcommand{\LineCommentCont}[1]{\Statex \hskip\ALG@thistlm%
  \parbox[t]{\dimexpr\linewidth-\ALG@thistlm}{\hangindent=\trianglerightwidth \hangafter=1 \strut$\triangleright$ #1\strut}}
\makeatother

% for footnote/marginnote
% see https://tex.stackexchange.com/a/133265/23098
\usepackage{tikz}
\newcommand{\circled}[1]{%
  \tikz[baseline=(char.base)]
  \node [draw, circle, inner sep = 0.5pt, font = \tiny, minimum size = 8pt] (char) {#1};
}
\renewcommand\thefootnote{\protect\circled{\arabic{footnote}}}

\newcommand{\score}[1]{{\bf [#1 分]}}

\newcommand{\rel}[1]{\xrightarrow{#1}}
\newcommand{\dstar}{\xRightarrow[]{\ast}}
\newcommand{\dplus}{\xRightarrow[]{+}}
\newcommand{\lm}{\xRightarrow[\text{lm}]{}}
\renewcommand{\rm}{\xRightarrow[\text{rm}]{}}
\newcommand{\dpluslm}{\xRightarrow[\text{lm}]{+}}
\newcommand{\dstarlm}{\xRightarrow[\text{lm}]{\ast}}
\newcommand{\dplusrm}{\xRightarrow[\text{rm}]{+}}
\newcommand{\dstarrm}{\xRightarrow[\text{rm}]{\ast}}

\newcommand{\sep}{\;\big\lvert\;}

\newcommand{\first}{\textsc{First}}
\newcommand{\follow}{\textsc{Follow}}

% see https://tex.stackexchange.com/a/109906/23098
\usepackage{empheq}
\newcommand*\widefbox[1]{\fbox{\hspace{2em}#1\hspace{2em}}} % feel free to modify this file if you understand LaTeX well
%%%%%%%%%%%%%%%%%%%%
\title{编译原理作业 (5)}
\me{\hspace{50pt}}{\hspace{70pt}}
\date{2022年12月08日}
%%%%%%%%%%%%%%%%%%%%
\begin{document}
\maketitle
%%%%%%%%%%%%%%%%%%%%
\noplagiarism % PLEASE DON'T DELETE THIS LINE!
%%%%%%%%%%%%%%%%%%%%
\begin{abstract}
\end{abstract}
%%%%%%%%%%%%%%%%%%%%
\beginrequired
%%%%%%%%%%%%%%%

\definecolor{dkgreen}{rgb}{0,0.6,0}
\definecolor{gray}{rgb}{0.5,0.5,0.5}
\definecolor{mauve}{rgb}{0.58,0,0.82}
\lstset{frame=tb,
	language=Java,
	aboveskip=3mm,
	belowskip=3mm,
	showstringspaces=false,
	columns=flexible,
	basicstyle = \ttfamily\small,
	numbers=none,
	numberstyle=\tiny\color{gray},
	keywordstyle=\color{blue},
	commentstyle=\color{dkgreen},
	stringstyle=\color{mauve},
	breaklines=true,
	breakatwhitespace=true,
	tabsize=3,
	flexiblecolumns,
	numbers=left,
	frame=lrtb
}

%%%%%%%%%%%%%%%
\begin{problem}
  现考虑为课堂上展示的 \texttt{Cymbol} 语言添加 \texttt{struct} 语法结构,
  以支持如下图所示的代码片段。
  \fig{width = 0.40\textwidth}{figs/struct-code}
  \begin{enumerate}[(1)]
    \item 请给出描述 \texttt{struct} 结构的文法。
    \item 请简述如何为新的 \texttt{Cymbol} 语言构建作用域树。
      可以只介绍增量部分,并只需给出要点,例如:
      \begin{itemize}
        \item 何时开始新的 \texttt{struct} 作用域?
        \item 何时退出 \texttt{struct} 作用域?
        \item 何时以及如何解析成员访问 $a.b.y$?
      \end{itemize}
  \end{enumerate}
\end{problem}

\begin{solution}
	\begin{enumerate}[(1)]
		\item
			在 Cymbol 文法基础上,struct 结构的文法描述如下
			\begin{lstlisting}
prog : (varDecl | functionDecl | structDecl)* EOF ;
type : 'int' | 'double' | 'void' | structType ;
structDecl : 'struct' ID '{' (varDecl | structDecl)* '}' ';' ;
structType :  ID ;
stat : structDecl | ... ;
expr : expr '.' ID | ... ;
			\end{lstlisting}
		\item
			\begin{itemize}
				\item enterStructDecl 开始新的 struct 作用域
				\item exitStructDecl 退出 struct 作用域
				\item struct 对应的 symbol 应置入其作用域的 enclosing 作用域中
				\item 解析结构体成员访问运算符时,应先在符号表中 resolve 结构体变量对应的 symbol 和 scope,然后在这个
				scope 中 resolve 结构体成员,该过程可递归进行
			\end{itemize}

	\end{enumerate}
\end{solution}
%%%%%%%%%%%%%%%

\pagebreak
%%%%%%%%%%%%%%%
\begin{problem}
  考虑如下图所示的变量声明文法,现要将每个 \texttt{id} 的名字和类型加入符号表
  (假设只有一个全局作用域)。请描述如何使用 ANTLR 4 为该任务编写监听器?
  \fig{width = 0.35\textwidth}{figs/type-decl}
\end{problem}

\begin{solution}
	将产生式改写为如下文法文件
	\begin{lstlisting}
D : T L
;
T : INT # TypeInt
	| FLOAT # TypeFloat
;
L : L COMMA ID # DecMore
	| ID # DecOne
;
	\end{lstlisting}
	\begin{lstlisting}
import org.antlr.v4.runtime.tree.ParseTreeProperty;
public class ExampleListener extends ExampleBaseListener {
	private final ParseTreeProperty<String> typeProperty = new ParseTreeProperty<>();
	@Override public void enterTypeInt(SysYParser.TypeIntContext ctx) {
		typeProperty.put(ctx.getParent(), "int");
	}
	@Override public void enterTypeFloat(SysYParser.TypeFloatContext ctx) {
		typeProperty.put(ctx.getParent(), "float");
	}
	@Override public void enterDecMore(SysYParser.DecMoreContext ctx) {
		System.err.printf("+ %s %s%n", typeProperty.get(ctx.getParent()),
		ctx.ID().getSymbol().getText());
		typeProperty.put(ctx, typeProperty.get(ctx.getParent()));
	}
	@Override public void enterDecOne(SysYParser.DecOneContext ctx) {
		System.err.printf("+ %s %s%n", typeProperty.get(ctx.getParent()),
		ctx.ID().getSymbol().getText());
	}
}
	\end{lstlisting}
	简单起见,使用字符串表示类型\\
	对于输入 int a, b, c ,对应的输出为
	\begin{lstlisting}
+ int c
+ int b
+ int a
	\end{lstlisting}
\end{solution}
%%%%%%%%%%%%%%%

%%%%%%%%%%%%%%%%%%%%
% 如果没有需要订正的题目,可以把这部分删掉

% \begincorrection
%%%%%%%%%%%%%%%%%%%%

%%%%%%%%%%%%%%%%%%%%
% 如果没有反馈,可以把这部分删掉
\beginfb

你可以写
\begin{itemize}
  \item 对课程及教师的建议与意见
  \item 教材中不理解的内容
  \item 希望深入了解的内容
  \item $\cdots$
\end{itemize}
%%%%%%%%%%%%%%%%%%%%
\end{document}