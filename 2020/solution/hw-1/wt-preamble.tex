%% geometry for A4 paper, word preference
\usepackage{geometry}
\geometry{
  left=20.0mm,
  right=20.mm,
  top=10mm,
  bottom=20.0mm
}
%% for font 
\usepackage{xeCJK}
%\newcommand{\song}{\CJKfamily{song}}
%\newcommand{\kai}{\CJKfamily{kai}}

%% author
\newcommand{\me}[2]{
  \author{
    {\bfseries 姓名:}\underline{#1}\hspace{2em}
    {\bfseries 邮箱:}\underline{#2}\hspace{2em}
  }
}

%% for picture
\usepackage{graphicx}

%% for math
\usepackage{amsmath,ntheorem}
\theoremstyle{break} %标题换行
%\theorembodyfont{\song}
%\theoremheaderfont{\kai\bfseries}
\newtheorem*{problem}{\Large{题目}}[subsection]

\newenvironment{solution}
{
	\textbf{\Large{解答:}}\\
}   

%% for pseudocode
\usepackage{algorithm}
\usepackage{algpseudocode}

\algblockdefx[SWITCH]{Switch}{EndSwitch}
  [1]{\bf{Switch :} #1}
  [0]{\bf{EndSwitch}}
\algblockdefx[CASE]{Case}{EndCase}%
  [1]{\bf{Case :} #1}%
  [0]{\bf{EndCase} }


%% algorithm分页
\makeatletter
\newenvironment{breakablealgorithm}
  {% \begin{breakablealgorithm}
    \begin{center}
      \refstepcounter{algorithm}% New algorithm
      \hrule height.8pt depth0pt \kern2pt% \@fs@pre for \@fs@ruled
      \renewcommand{\caption}[2][\relax]{% Make a new \caption
        {\raggedright\textbf{\ALG@name~\thealgorithm} ##2\par}%
        \ifx\relax##1\relax % #1 is \relax
          \addcontentsline{loa}{algorithm}{\protect\numberline{\thealgorithm}##2}%
        \else % #1 is not \relax
          \addcontentsline{loa}{algorithm}{\protect\numberline{\thealgorithm}##1}%
        \fi
        \kern2pt\hrule\kern2pt
      }
  }{% \end{breakablealgorithm}
      \kern2pt\hrule\relax% \@fs@post for \@fs@ruled
    \end{center}
  }
\makeatother
  

